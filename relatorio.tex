\documentclass[a4paper]{scrartcl}

\usepackage[brazil]{babel}
\usepackage[utf8]{inputenc}
\usepackage{amsmath}
\usepackage{graphicx}
%\usepackage[colorinlistoftodos]{todonotes}
\usepackage{pgf,tikz}
\usetikzlibrary{arrows,automata}
\usepackage{hyperref}
\usepackage{float}
\usepackage{listings}
\usepackage{booktabs}
\usepackage{amsfonts}
\usepackage{multirow}

\title{Relatório da Fase 2 do Projeto de MAC5784}
\subtitle{LuaMan: um clone de Pac-Man}

\author{Ricardo Ferreira Guimarães \\ 7577650}

\date{\today}

\begin{document}
\maketitle

\section{Introdução}

Neste projeto programa-se um jogo similar ao Pac-Man para estudo de aplicação de
técnicas de inteligência artificial em jogos de computador. Nesta fase em particular,
o objetivo consiste em tornar os fantasmas agentes inteligentes, isto é, fazê-los terem um comportamento estratégico, em especial de forma emergente.

Para este fim, tem-se a intenção de utilizar conceitos vistos durante o curso a fim
de construir e avaliar tal comportamento.

\section{Algoritmos e Técnicas Empregados}

Dentre as técnicas vistas no curso, optou-se por utilizar autômatos finitos,
por sua simplicidade e fácil adequação ao problema. Tentou-se utilizar árvores de
comportamento, mas estas exigiam uma complexidade maior na implementação e modelagem mais
sofisticada, que talvez, não fosse adicionar qualidade compensatório no objetivo final.
Também, explorou-se a utilização de árvores de decisão, porém estas iam aos poucos se tornando
muito extensas e de difícil integração com o código existente.

Já para os fantasmas as ideias de \textit{heat map} e autômato celular não foram empregadas
por dificuldade em encaixá-las no modelo já criado para os atores.

Do ponto de vista do LuaMan, o objetivo final do brinquedo desenvolvido é que o
personagem principal coma todas pílulas, sem ser pego pelos fantasmas, e em segundo plano, que coma alguns fantasmas quando for seguro ou conveniente. 
Para este fim, foram pensados três estados de comportamento, que trabalham em conjunto para atingir o objetivo final:

\begin{itemize}
	\item \textit{eat:} Neste estado, o Pac-Man tem por objetivo principal comer os itens do mapa.
	\item \textit{run:} Estado em que o Pac-Man foge dos fantasmas ``ativos''.
	\item \textit{hunt:} Quando o Pac-Man come um pílula de poder e passa a caçar os fantasmas.
\end{itemize}

Em todos os estados utiliza-se o algoritmo A* para formular um rota (ou plano) para o LuaMan,
o que muda entre eles são: a função heurística, a função de avaliação de um nó e a verificação
de objetivo. Quando o Pac-Man atinge um objetivo ou muda de estado gera-se um novo plano, de acordo como estado do autômato.

Já do ponto de vista dos fantasmas, seu único objetivo é o de capturar o LuaMan. Para que não acontecesse de todos eles ficarem sempre juntos
e perseguindo o LuaMan por um mesmo caminho sempre e para que tivessem um comportamento que pode ser considerado inteligente,
foram pensados quatro estados possíveis para este tipo de entidade:

\begin{itemize}
	\item \textit{wander:} Neste estado, o fantasma vai para um ponto qualquer do mapa.
	\item \textit{seek:} Estado no qual o fantasma persegue o LuaMan.
	\item \textit{avoid:} Quando o LuaMan come um pílula de poder, os fantasmas entram em ``pânico'' e deixam de perseguir o LuaMan.
	\item \textit{restore:} Após ser comido pelo LuaMan, um fantasma deve retornar ao centro do mapa para se recuperar.
\end{itemize}

\subsection{Algoritmo A*}

Para guiar o LuaMan e os fantasmas até seus variados objetivos, escolheu-se implementar um A* relativamente simples. Neste algoritmo, faz-se uma busca expandindo os nós de menor custo da origem, até algum
objetivo (podem ser um ou mais nós). Ele faz isso avaliando o custo determinístico (gasto até o momento) e o custo esperado até o objetivo (custo heurístico).

Além de ser comumente utilizado em jogos e ser  de implementação consideravelmente simples
(mesmo tendo que desenvolver as próprias filas de prioridades), optou-se por este algoritmo por
poder incluir valores a mais nos cálculos dos custos, como será explicitado posteriormente para
cada estado dos autômatos que controlam o comportamento do LuaMan e dos fantasmas.

\section{Comportamento do LuaMan}

\subsection{Estado \textit{Eat}}

Neste estado o objetivo é bem simples: chegar à pílula (simples ou de poder), mais próxima.
Assim, considera-se um nó (ladrilho do jogo), um destino, se ele contém alguma pílula.

Entretanto, ao utilizar o A*, não encontrou-se uma heurística adequada (lembrando
que a função básica de avaliação do custo é f(n) = g(n) + h(n), onde h(n) é um heurística,
preferencialmente admissível). Isto pois, em um dado estado, teria de se saber como está a
distribuição de pílulas no mapa, o que poderia ser muito custoso para ser calculado em
um ciclo do jogo. Por estas razões a heurística básica é h(n) = 0.

Para fazer com que o Pac-Man evitasse os fantasmas foi adicionado um ``fator de risco'',
calculado pela fórmula $dangerFactor(n) =  \frac{\alpha_{eat}}{\text{distância mínima até um fantasma ativo}}$. Atualmente, $\alpha_{eat} = 100$ (quanto maior, mais distância o LuaMan
tentará manter dos fantasmas).

E com objetivo de diferenciar os nós pelo conteúdo, adiciona-se mais custo: ``custo de conteúdo'',
ele vale 0, se a célula contém uma pílula comum, 5 para pílula de energia e 2 para piso vazio; isto
faz com que o A*, ao tentar minimizar custos, produza caminhos com mais pílulas, mas com certa economia da pílula de energia.

Assim, para o A* define-se a função heurística neste estado por:
$h_{eat}(n) = h(n) + dangerFactor(n) + contentCost(n)$

Se um fantasma ativo chega a menos de uma distância de 8 unidades (norma L1), muda-se
para o estado \textit{run}. Ao pegar a pílula de poder, muda-se para o estado \textit{hunt}.

\subsection{Estado \textit{Run}}

Neste estado quer-se apenas fugir de todos os fantasmas. Em contraste com o estado anterior,
ainda querendo aproveitar o A*, há várias funções heurísticas possíveis para se avaliar um nó:
distância mínima até um fantasma, distância média até os fantasmas, fator de ramificação, entre outras. Entretanto, não encontrou-se uma função simples que caracterizasse um nó objetivo, as estratégias relativamente sofisticadas envolviam calcular o valor de distâncias para 
cada um dos nós, e aqueles com valor máximo seriam objetivo, mas isto poderia se mostrar custoso
computacionalmente. 

Assim, usando a geometria do mapa, foram fixados 5 pontos, chamados ``pontos de fuga'', são eles:
os quatro cantos e o centro do mapa. Nos testes, estes se mostraram suficiente para que o LuaMan
possa escolher para onde fugir, escolhendo, a cada iteração, o melhor entre eles (neste caso, aquele
que minimize a distância média até os fantasmas).

Até a fase 2, a função objetivo verificava se o nó era o melhor ponto de fuga, agora ela verifica se todos os fantasmas estão a pelo menos 8
unidades de distância do nó. Isto fez com que o LuaMan fosse mais flexível em sua fuga e evitava movimentos perigosos quando tentava-se mover
em direção a um ponto de fuga.

Para melhorar a capacidade de fuga, a cada movimento do LuaMan neste estado o plano inteiro é recalculado.
 Mesmo assim, a eficiência do jogo não foi demasiadamente comprometida (não houveram
diferenças perceptíveis na taxa de quadros).

Assim como no estado \textit{eat}, este também conta com um ``fator de risco'':

$dangerFactor(n) =  \frac{\alpha_{run}}{\text{distância mínima até um fantasma ativo}}$

(atualmente $\alpha_{run} = 1500$); e um ``custo de conteúdo'': valendo 2 para pílula comum ou piso vazio, e 0 para pílula de poder (isto para simular uma ``intenção de contra-ataque'', caso uma boa oportunidade surja).

Assim, para o A* define-se a função heurística neste estado por:

$h_{run}(n) = \min{L1(n, goals)} + dangerFactor(n) + contentCost(n)$

Deste estado, vai-se para \textit{eat}, se os fantasmas ficarem a mais de 10 unidades de distância (norma L1) e vai para \textit{hunt} ao pegar uma pílula de poder.

\subsection{Estado \textit{Hunt}}

Neste estado, sabidamente, o LuaMan está invulnerável e seu objetivo é caçar fantasmas, até que sua
energia esteja em um nível baixo (20\%, atualmente). Felizmente, tanto a função de verificação
de objetivo, quanto a função heurística foram facilmente determinadas:
uma célula é objetivo se contém um fantasma ativo, e a heurística é a distância (norma L1) até o fantasma mais próximo.

Neste caso, não faz sentido tem um ``fator de perigo'', mas ainda assim, utiliza-se o ``custo de conteúdo'' para priorizar caminhos com pílulas e evitar pegar outra pílula de energia, enquanto a
o nível de poder ainda está alto. Portanto, os custos escolhidos foram: 1 para piso ou pílula comum e 7 para pílula de energia.

Assim, para o A* define-se a função heurística neste estado por:

$h_{run}(n) = \min{L1(n, \text{fantasmas ativos}} + contentCost(n)$

\section{Comportamento dos Fantasmas}

\subsection{Estado \textit{Wander}}

Neste estado, cada fantasma escolhe um ponto aleatório livre no mapa de forma independente.
Cada um calcula um plano usando o A*. Como função heurística a norma L1 (distância de Manhattan)
até o destino escolhido. Se o destino for atingido, um novo ponto é selecionado da mesma forma.
A função que verifica o objetivo resume-se a verificar se o nó avaliado corresponde ao destino do
fantasma.

Um fantasma para de andar aleatoriamente se: chegar perto suficiente do LuaMan
(pois então passará a persegui-lo) ou se o LuaMan pegar uma pílula de poder.

\subsection{Estado \textit{Seek}}

Quando um fantasma está neste estado, calcula um plano até um ponto próximo do LuaMan.
A função heurística consiste na norma L1 do nó até o LuaMan. Para evitar que todos os fantasmas
se aglomerassem em um único caminho, dois mecanismos foram implementados, um deles (o outro será explicado no final da sessão) consiste em uma
relaxação no objetivo. Em vez de considerar um nó como final, se ele conter o LuaMan, a cada chamada
da função objetivo escolhe-se um número de 1 à 5, se o nó estiver a esta distância sorteada, ou menos
do LuaMan, então é considerado um nó objetivo.

O fantasma para de perseguir o LuaMan quando eles se distanciam muito, ou se o LuaMan pega uma pílula de poder.

\subsection{Estado \textit{Avoid}}

Um fantasma permanece neste estado enquanto o LuaMan estiver sob efeito da pílula de energia.
No Pac-Man original, quando o personagem principal come uma pílula de energia os fantasmas saem
pelo mapa de forma aleatória. Para simular o mesmo feito, este estado atualmente se comporta como
o \textit(wander), produzindo bons resultados.

\subsection{Estado \textit{Restore}}

Se um fantasma for comido pelo LuaMan, entrará neste estado, cujo objetivo se resume a voltar ao
centro do mapa, onde poderá se recuperar. Neste caso, a verificação de objetivo e heurística são triviais: o objetivo é chegar em algum quadradinho da região central, onde a maior parte dos fantasmas aparece e a heurística é simplesmente a distância de Manhattan até o centro do mapa.

\subsection{Observações}

Além do custo determinístico e das funções heurísticas, adiciona-se também ao custo do
nó um fator de custo proporcional à proximidade do nó aos demais fantasmas. Isto é feito
para que o algoritmo preferencie caminhos que aumentem a distância entre os fantasmas, evitando
aglomerações.

Nesta fase, utiliza-se a seguinte fórmula como definição do custo (com $\alpha_{prox} = 1000$ ):

$proximityFactor(n) =  \frac{\alpha_{prox}}{\text{{(soma das normas L1 até os outros fantasmas})}^2}$

\section{Tecnologias}

Para construção deste jogo utiliza-se a biblioteca löve2d (\url{http://love2d.org}),
a qual auxilia o desenvolvimento de jogos utilizando a linguagem Lua.

\section{Funcionamento}

\subsection{Personagem principal}

%Nesta fase, LuaMan, o personagem principal é controlado pelo jogador utilizando as setas do teclado.
Nesta fase o personagem principal é controlado por um autômato finito determinístico.
O código responsável está no arquivo \texttt{player.lua}.

\subsection{Inimigos}

Assim como o LuaMan, essas entidades também são controladas automaticamente por uma autômato finito determinístico,
como explicado anteriormente. A lógica dos fantasmas está implementada no arquivo \texttt{enemy.lua}, em particular na função
\texttt{enemy.Enemy:act()}

\section{Execução}

Para executar o jogo, basta extrair o pacote referente ao sistema operacional desejado e executar o
arquivo correspondente (luaman.exe no Windows, run.sh no Linux e luaman.app no MacOS).

\section{Observações}

O código do jogo está disponível em: \url{https://github.com/rfguimaraes/LuaMan}.
Pacotes para cada sistema operacional estão disponíveis em: \url{https://github.com/rfguimaraes/LuaMan/releases/tag/Fase2}.

\section{Considerações Finais}

%Por meio desta etapa foi possível entender melhor a aplicabilidade das diferentes técnicas de
%tomada de decisão. Graças ao modelo flexível adotado, pode-se ajustar facilmente o comportamento
%do Pac-Man, trocando alguns valores (parâmetros das heurísticas, por exemplo).
%Além disso, foi possível lidar com problemas de integração entre o código existente de movimentação
%e o controlador recém-desenvolvido (etapa que demandou bastante tempo).
%
%Entre as melhorias que poderiam ser feitas estão: dar prioridade à direção na qual já está se movendo
%(evitando mudanças de direção desnecessárias), mudar a heurística do estado \textit{hunt} para que o Pac-Man seja mais ou menos agressivo, de forma proporcional à energia que ainda lhe resta e pensar
%em formas mais eficientes de decidir a movimentação do Pac-Man, fazendo pré-processamento do mapa
%ou limitando a análise apenas dos objetos até uma certa distância dele.

Nesta fase, verificou-se a dificuldade de estabelecer uma coordenação entre agentes,
principalmente quando quer se restringir a comunicação, em busca de um comportamento emergente
como acontece muitas vezes no mundo real.

Também nota-se que os fantasmas, apesar de não estarem realmente programados com um conceito de time,
onde se comunicam, delegam papéis e se diferenciam entre si, dão a impressão de que atuam conjuntamente
para atingir seu objetivo.

Como detalhes técnicos, foram corrigidos diversas falhas relacionadas à movimentação e execução do plano.
Em particular algumas delas ocorriam quando o jogo passava muito tempo no autômato ou no A*. Este evento
ressalta o aspecto de que a solução de inteligência tem como um requisito funcional o tempo limitado
entre dois frames de animação (de forma geral).

Entre melhorias que ainda poderiam ser feitas estão: estabelecer uma estratégia mais bem definida para os fantasmas,
fazer o LuaMan considerar mais aspectos sobre o ambiente (se está na zona de passagem livre para fantasmas, se está perto
do ponto de restauração no centro do mapa, etc.), substituir o A* por algum outro algoritmo que não refaça todos os cálculos
a partir do zero quando refaz um plano (por exemplo, o D* lite).

\end{document}