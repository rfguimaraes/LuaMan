\documentclass[a4paper]{scrartcl}

\usepackage[brazil]{babel}
\usepackage[utf8]{inputenc}
\usepackage{amsmath}
\usepackage{graphicx}
%\usepackage[colorinlistoftodos]{todonotes}
\usepackage{pgf,tikz}
\usetikzlibrary{arrows,automata}
\usepackage{hyperref}
\usepackage{float}
\usepackage{listings}
\usepackage{booktabs}
\usepackage{amsfonts}
\usepackage{multirow}

\title{Relatório da Fase 1 do Projeto de MAC5784}
\subtitle{LuaMan: um clone de Pac-Man}

\author{Ricardo Ferreira Guimarães \\ 7577650}

\date{\today}

\begin{document}
\maketitle

\section{Introdução}

Neste projeto programa-se um jogo similar ao Pac-Man para estudo de aplicação de
técnicas de inteligência artificial em jogos de computador.

\section{Tecnologias}

Para construção deste jogo utiliza-se a biblioteca löve2d (\url{http://love2d.org}),
a qual auxilia o desenvolvimento de jogos utilizando a linguagem Lua.

\section{Funcionamento}

\subsection{Personagem principal}

Nesta fase, LuaMan, o personagem principal é controlado pelo jogador utilizando as setas do teclado.
O código responsável está no arquivo \texttt{player.lua}.

\subsection{Inimigos}

Os inimigos, representados pelos 4 fantasmas, se movem aleatoriamente da seguinte maneira:
cada fantasma mantem a mesma direção na qual tentava andar com 25\% de chance, e com os outros 75\%
escolhia uniformemente alguma direção (para cima, para baixo, para esquerda ou para direita), mesmo
que ainda assim pudesse se manter na mesma direção que no passo anterior.

A lógica dos fantasmas está implementada no arquivo \texttt{enemy.lua}, em particular na função
\texttt{enemy.Enemy:act()}

\section{Execução}

Na pasta \texttt{releases}, se encontram arquivos compactados para cada sistema operacional disponível.
Para executar o jogo, basta extrair o pacote referente ao sistema operacional desejado e executar o
arquivo correspondente (luaman.exe no Windows, run.sh no Linux e luaman.app no MacOS).

\section{Observações}

O código do jogo está disponível em: \url{https://github.com/rfguimaraes/LuaMan}.
Pacotes para cada sistema operacional estão disponíveis em: \url{https://github.com/rfguimaraes/LuaMan/releases/tag/Fase1}.

\end{document}